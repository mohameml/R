% Options for packages loaded elsewhere
\PassOptionsToPackage{unicode}{hyperref}
\PassOptionsToPackage{hyphens}{url}
%
\documentclass[
]{article}
\usepackage{amsmath,amssymb}
\usepackage{iftex}
\ifPDFTeX
  \usepackage[T1]{fontenc}
  \usepackage[utf8]{inputenc}
  \usepackage{textcomp} % provide euro and other symbols
\else % if luatex or xetex
  \usepackage{unicode-math} % this also loads fontspec
  \defaultfontfeatures{Scale=MatchLowercase}
  \defaultfontfeatures[\rmfamily]{Ligatures=TeX,Scale=1}
\fi
\usepackage{lmodern}
\ifPDFTeX\else
  % xetex/luatex font selection
\fi
% Use upquote if available, for straight quotes in verbatim environments
\IfFileExists{upquote.sty}{\usepackage{upquote}}{}
\IfFileExists{microtype.sty}{% use microtype if available
  \usepackage[]{microtype}
  \UseMicrotypeSet[protrusion]{basicmath} % disable protrusion for tt fonts
}{}
\makeatletter
\@ifundefined{KOMAClassName}{% if non-KOMA class
  \IfFileExists{parskip.sty}{%
    \usepackage{parskip}
  }{% else
    \setlength{\parindent}{0pt}
    \setlength{\parskip}{6pt plus 2pt minus 1pt}}
}{% if KOMA class
  \KOMAoptions{parskip=half}}
\makeatother
\usepackage{xcolor}
\usepackage[margin=1in]{geometry}
\usepackage{color}
\usepackage{fancyvrb}
\newcommand{\VerbBar}{|}
\newcommand{\VERB}{\Verb[commandchars=\\\{\}]}
\DefineVerbatimEnvironment{Highlighting}{Verbatim}{commandchars=\\\{\}}
% Add ',fontsize=\small' for more characters per line
\usepackage{framed}
\definecolor{shadecolor}{RGB}{248,248,248}
\newenvironment{Shaded}{\begin{snugshade}}{\end{snugshade}}
\newcommand{\AlertTok}[1]{\textcolor[rgb]{0.94,0.16,0.16}{#1}}
\newcommand{\AnnotationTok}[1]{\textcolor[rgb]{0.56,0.35,0.01}{\textbf{\textit{#1}}}}
\newcommand{\AttributeTok}[1]{\textcolor[rgb]{0.13,0.29,0.53}{#1}}
\newcommand{\BaseNTok}[1]{\textcolor[rgb]{0.00,0.00,0.81}{#1}}
\newcommand{\BuiltInTok}[1]{#1}
\newcommand{\CharTok}[1]{\textcolor[rgb]{0.31,0.60,0.02}{#1}}
\newcommand{\CommentTok}[1]{\textcolor[rgb]{0.56,0.35,0.01}{\textit{#1}}}
\newcommand{\CommentVarTok}[1]{\textcolor[rgb]{0.56,0.35,0.01}{\textbf{\textit{#1}}}}
\newcommand{\ConstantTok}[1]{\textcolor[rgb]{0.56,0.35,0.01}{#1}}
\newcommand{\ControlFlowTok}[1]{\textcolor[rgb]{0.13,0.29,0.53}{\textbf{#1}}}
\newcommand{\DataTypeTok}[1]{\textcolor[rgb]{0.13,0.29,0.53}{#1}}
\newcommand{\DecValTok}[1]{\textcolor[rgb]{0.00,0.00,0.81}{#1}}
\newcommand{\DocumentationTok}[1]{\textcolor[rgb]{0.56,0.35,0.01}{\textbf{\textit{#1}}}}
\newcommand{\ErrorTok}[1]{\textcolor[rgb]{0.64,0.00,0.00}{\textbf{#1}}}
\newcommand{\ExtensionTok}[1]{#1}
\newcommand{\FloatTok}[1]{\textcolor[rgb]{0.00,0.00,0.81}{#1}}
\newcommand{\FunctionTok}[1]{\textcolor[rgb]{0.13,0.29,0.53}{\textbf{#1}}}
\newcommand{\ImportTok}[1]{#1}
\newcommand{\InformationTok}[1]{\textcolor[rgb]{0.56,0.35,0.01}{\textbf{\textit{#1}}}}
\newcommand{\KeywordTok}[1]{\textcolor[rgb]{0.13,0.29,0.53}{\textbf{#1}}}
\newcommand{\NormalTok}[1]{#1}
\newcommand{\OperatorTok}[1]{\textcolor[rgb]{0.81,0.36,0.00}{\textbf{#1}}}
\newcommand{\OtherTok}[1]{\textcolor[rgb]{0.56,0.35,0.01}{#1}}
\newcommand{\PreprocessorTok}[1]{\textcolor[rgb]{0.56,0.35,0.01}{\textit{#1}}}
\newcommand{\RegionMarkerTok}[1]{#1}
\newcommand{\SpecialCharTok}[1]{\textcolor[rgb]{0.81,0.36,0.00}{\textbf{#1}}}
\newcommand{\SpecialStringTok}[1]{\textcolor[rgb]{0.31,0.60,0.02}{#1}}
\newcommand{\StringTok}[1]{\textcolor[rgb]{0.31,0.60,0.02}{#1}}
\newcommand{\VariableTok}[1]{\textcolor[rgb]{0.00,0.00,0.00}{#1}}
\newcommand{\VerbatimStringTok}[1]{\textcolor[rgb]{0.31,0.60,0.02}{#1}}
\newcommand{\WarningTok}[1]{\textcolor[rgb]{0.56,0.35,0.01}{\textbf{\textit{#1}}}}
\usepackage{graphicx}
\makeatletter
\def\maxwidth{\ifdim\Gin@nat@width>\linewidth\linewidth\else\Gin@nat@width\fi}
\def\maxheight{\ifdim\Gin@nat@height>\textheight\textheight\else\Gin@nat@height\fi}
\makeatother
% Scale images if necessary, so that they will not overflow the page
% margins by default, and it is still possible to overwrite the defaults
% using explicit options in \includegraphics[width, height, ...]{}
\setkeys{Gin}{width=\maxwidth,height=\maxheight,keepaspectratio}
% Set default figure placement to htbp
\makeatletter
\def\fps@figure{htbp}
\makeatother
\setlength{\emergencystretch}{3em} % prevent overfull lines
\providecommand{\tightlist}{%
  \setlength{\itemsep}{0pt}\setlength{\parskip}{0pt}}
\setcounter{secnumdepth}{-\maxdimen} % remove section numbering
\ifLuaTeX
  \usepackage{selnolig}  % disable illegal ligatures
\fi
\IfFileExists{bookmark.sty}{\usepackage{bookmark}}{\usepackage{hyperref}}
\IfFileExists{xurl.sty}{\usepackage{xurl}}{} % add URL line breaks if available
\urlstyle{same}
\hypersetup{
  hidelinks,
  pdfcreator={LaTeX via pandoc}}

\author{}
\date{\vspace{-2.5em}}

\begin{document}

\hypertarget{correction-du-exo}{%
\section{Correction du Exo :}\label{correction-du-exo}}

\hypertarget{chargement-du-data}{%
\subsubsection{1. chargement du data :}\label{chargement-du-data}}

\begin{Shaded}
\begin{Highlighting}[]
\CommentTok{\# data :}

\NormalTok{variete}\OtherTok{\textless{}{-}}\FunctionTok{rep}\NormalTok{(}\DecValTok{1}\SpecialCharTok{:}\DecValTok{6}\NormalTok{,}\FunctionTok{c}\NormalTok{(}\DecValTok{5}\NormalTok{,}\DecValTok{5}\NormalTok{,}\DecValTok{5}\NormalTok{,}\DecValTok{5}\NormalTok{,}\DecValTok{5}\NormalTok{,}\DecValTok{5}\NormalTok{))}

\NormalTok{vitamine}\OtherTok{\textless{}{-}}\FunctionTok{c}\NormalTok{(}\FloatTok{93.6}\NormalTok{,}\FloatTok{95.3}\NormalTok{,}\DecValTok{96}\NormalTok{,}\FloatTok{93.7}\NormalTok{,}\FloatTok{96.2}\NormalTok{,}\FloatTok{95.3}\NormalTok{,}\FloatTok{96.9}\NormalTok{,}\FloatTok{95.8}\NormalTok{,}\FloatTok{97.3}\NormalTok{,}\FloatTok{97.7}\NormalTok{,}\FloatTok{94.5}\NormalTok{,}\DecValTok{97}\NormalTok{,}\FloatTok{97.8}\NormalTok{,}\DecValTok{97}\NormalTok{,}
            \FloatTok{98.3}\NormalTok{,}\FloatTok{98.8}\NormalTok{,}\FloatTok{98.2}\NormalTok{,}\FloatTok{97.8}\NormalTok{,}\FloatTok{97.2}\NormalTok{,}\FloatTok{97.9}\NormalTok{,}\FloatTok{94.6}\NormalTok{,}\FloatTok{97.8}\NormalTok{,}\DecValTok{98}\NormalTok{,}\DecValTok{95}\NormalTok{,}\FloatTok{98.9}\NormalTok{,}\FloatTok{93.2}\NormalTok{,}\FloatTok{94.4}\NormalTok{,}\FloatTok{93.8}\NormalTok{,}\FloatTok{95.6}\NormalTok{,}\FloatTok{94.8}\NormalTok{)}

\NormalTok{variete }\OtherTok{\textless{}{-}} \FunctionTok{factor}\NormalTok{(variete)}

\NormalTok{df }\OtherTok{\textless{}{-}} \FunctionTok{data.frame}\NormalTok{(variete , vitamine)}
\FunctionTok{head}\NormalTok{(df)}
\end{Highlighting}
\end{Shaded}

\begin{verbatim}
##   variete vitamine
## 1       1     93.6
## 2       1     95.3
## 3       1     96.0
## 4       1     93.7
## 5       1     96.2
## 6       2     95.3
\end{verbatim}

\begin{Shaded}
\begin{Highlighting}[]
\FunctionTok{View}\NormalTok{(df)}
\end{Highlighting}
\end{Shaded}

\hypertarget{modele-anova-1-facteur-uxe0-effets-fixes}{%
\subsubsection{2. Modele ANOVA 1 facteur à effets fixes
:}\label{modele-anova-1-facteur-uxe0-effets-fixes}}

\begin{Shaded}
\begin{Highlighting}[]
\CommentTok{\# modele :}

\NormalTok{modele }\OtherTok{\textless{}{-}} \FunctionTok{aov}\NormalTok{(vitamine }\SpecialCharTok{\textasciitilde{}}\NormalTok{ variete , }\AttributeTok{data =}\NormalTok{ df)}
\FunctionTok{summary}\NormalTok{(modele)}
\end{Highlighting}
\end{Shaded}

\begin{verbatim}
##             Df Sum Sq Mean Sq F value  Pr(>F)   
## variete      5  45.84   9.167   5.713 0.00131 **
## Residuals   24  38.51   1.605                   
## ---
## Signif. codes:  0 '***' 0.001 '**' 0.01 '*' 0.05 '.' 0.1 ' ' 1
\end{verbatim}

\hypertarget{verification-de-hypothuxe9se-de-modele}{%
\subsubsection{3. verification de hypothése de
modele}\label{verification-de-hypothuxe9se-de-modele}}

\hypertarget{hypthuxe9s-de-normalituxe9}{%
\paragraph{hypthés de normalité :}\label{hypthuxe9s-de-normalituxe9}}

\begin{itemize}
\tightlist
\item
  avce QQ-plot :
\end{itemize}

\begin{Shaded}
\begin{Highlighting}[]
\DocumentationTok{\#\#\# ==== qq{-}plot}

\FunctionTok{qqnorm}\NormalTok{(}\FunctionTok{resid}\NormalTok{(modele))}
\FunctionTok{qqline}\NormalTok{(}\FunctionTok{resid}\NormalTok{(modele))}
\end{Highlighting}
\end{Shaded}

\includegraphics{solution_files/figure-latex/unnamed-chunk-3-1.pdf}

\begin{itemize}
\tightlist
\item
  avce histogramme :
\end{itemize}

\begin{Shaded}
\begin{Highlighting}[]
\FunctionTok{hist}\NormalTok{(}\FunctionTok{resid}\NormalTok{(modele))}
\end{Highlighting}
\end{Shaded}

\includegraphics{solution_files/figure-latex/unnamed-chunk-4-1.pdf}

\begin{itemize}
\tightlist
\item
  Avce test de shapiro :
\end{itemize}

\begin{Shaded}
\begin{Highlighting}[]
\FunctionTok{shapiro.test}\NormalTok{(}\FunctionTok{resid}\NormalTok{(modele))}
\end{Highlighting}
\end{Shaded}

\begin{verbatim}
## 
##  Shapiro-Wilk normality test
## 
## data:  resid(modele)
## W = 0.9563, p-value = 0.2485
\end{verbatim}

on a un p-value = 24\% \textgreater{} 5\% donc on conserve l'hypothsée
de normalité de errures \(\epsilon_{i,j}\)

\hypertarget{verification-de-homoguxe9nite-des-varinace-de-ruxe9sidus-avce-le-test-de-bartlett}{%
\paragraph{verification de homogénite des varinace de résidus avce le
test de Bartlett
:}\label{verification-de-homoguxe9nite-des-varinace-de-ruxe9sidus-avce-le-test-de-bartlett}}

\begin{Shaded}
\begin{Highlighting}[]
\FunctionTok{bartlett.test}\NormalTok{(}\FunctionTok{resid}\NormalTok{(modele)}\SpecialCharTok{\textasciitilde{}}\NormalTok{variete , }\AttributeTok{data =}\NormalTok{ df)}
\end{Highlighting}
\end{Shaded}

\begin{verbatim}
## 
##  Bartlett test of homogeneity of variances
## 
## data:  resid(modele) by variete
## Bartlett's K-squared = 5.6023, df = 5, p-value = 0.3469
\end{verbatim}

on a p-value = 34\% \textgreater{} 5\% donc on garde l'hypothése
d'homogénite des variances des residus

\hypertarget{resulats-du-moduxe9le}{%
\subsubsection{4. Resulats du Modéle :}\label{resulats-du-moduxe9le}}

\begin{Shaded}
\begin{Highlighting}[]
\CommentTok{\# modele :}
\NormalTok{modele}
\end{Highlighting}
\end{Shaded}

\begin{verbatim}
## Call:
##    aov(formula = vitamine ~ variete, data = df)
## 
## Terms:
##                 variete Residuals
## Sum of Squares   45.836    38.512
## Deg. of Freedom       5        24
## 
## Residual standard error: 1.266754
## Estimated effects may be unbalanced
\end{verbatim}

\begin{Shaded}
\begin{Highlighting}[]
\CommentTok{\# la fonction summary :}
\FunctionTok{summary}\NormalTok{(modele)}
\end{Highlighting}
\end{Shaded}

\begin{verbatim}
##             Df Sum Sq Mean Sq F value  Pr(>F)   
## variete      5  45.84   9.167   5.713 0.00131 **
## Residuals   24  38.51   1.605                   
## ---
## Signif. codes:  0 '***' 0.001 '**' 0.01 '*' 0.05 '.' 0.1 ' ' 1
\end{verbatim}

\begin{itemize}
\item
  on a un p-value \textless{} 5\% donc on peut réjetre l'hypothsé H0
  donc la variéte de pomme à un effet sur la teneur de vutamine C en
  pommme
\item
  on a que la carée moyenne de residue \(CM_r\) = 1.605 est une
  estimation de \(\sigma^2\)
\end{itemize}

\end{document}
